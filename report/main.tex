
\documentclass{acm_proc_article-sp-sigmod09}

\usepackage{graphicx} 
\usepackage[english]{babel}
\usepackage{cleveref}
\usepackage{float}
\usepackage{subfig}

\begin{document}



%
% --- Author Metadata here ---
%\conferenceinfo{ACM SIGMOD}{'09 Providence, RI, USA}
%\setpagenumber{50}
%\CopyrightYear{2002} % Allows default copyright year (2002) to be over-ridden - IF NEED BE.
%\crdata{0-12345-67-8/90/01}  % Allows default copyright data (X-XXXXX-XX-X/XX/XX) to be over-ridden.
% --- End of Author Metadata ---

\title{A Sample {\ttlit ACM} SIG Proceedings Paper in LaTeX
Format\titlenote{(Produces the permission block, copyright information and page numbering). For use with ACM\_PROC\_ARTICLE-SP.CLS V2.6SP. Supported by ACM.}}
%
% You need the command \numberofauthors to handle the "boxing"
% and alignment of the authors under the title, and to add
% a section for authors number 4 through n.

\numberofauthors{2}

\author{
	Willi Menapace\\
	\texttt{203778}\\
	\texttt{willi.menapace@studenti.unitn.it}
	\and
	Luca Zanella\\
	\texttt{207520}\\
	\texttt{luca.zanella-3@studenti.unitn.it}
}

%Intro
%Data Acquisition and compression
%Data transofrmation
%Data cleaning
%Preliminary analysis
%%Introduzione di come funzionano i taxi a new york in generale eg yellow e green. Divisione della citta' in zone. Come funziona la tariffazione. Tariffe speciali per aeroporti, fare vedere mappa di dove sono gli aeroporti
%%Descrizione delle cose piu' interessanti attributo per attributo. Partiamo dal volume totale dei taxi e di come sia cambiato negli anni. Facciamo vedere i guadagni e la distribuzione negli orari e nei distretti (Facendo i punti obbligatodi dell'assignment a parte il clustering)
%%Descrizione di tutte le cose interessanti attributo per attributo
%Traffic segmentation
%%Selezione degli attributi per il clustering, selezione di k. Commenti sui cluster ottenuti. Visualizzazione grafo.
%Traffic flow analysis (Moviemnti tra le varie zone)
%Yellow vs Green
%Airport traffic analysis
%Zones classification
%
% Appendices
%
% How to run it
% Bonus section spark cluster



\maketitle
\begin{abstract}
This paper provides a sample of a \LaTeX\ document which conforms to
the formatting guidelines for ACM SIG Proceedings.
It complements the document \textit{Author's Guide to Preparing
ACM SIG Proceedings Using \LaTeX$2_\epsilon$\ and Bib\TeX}. This
source file has been written with the intention of being
compiled under \LaTeX$2_\epsilon$\ and BibTeX.

The developers have tried to include every imaginable sort
of ``bells and whistles", such as a subtitle, footnotes on
title, subtitle and authors, as well as in the text, and
every optional component (e.g. Acknowledgments, Additional
Authors, Appendices), not to mention examples of
equations, theorems, tables and figures.

To make best use of this sample document, run it through \LaTeX\
and BibTeX, and compare this source code with the printed
output produced by the dvi file.
\end{abstract}

\section{Introduction}
The NYC Taxi and Limousine Commission (TLC) has publicly made available a dataset containing data about taxi trips performed in New York from January 2009 to June 2018. This study aims at exploiting this dataset to understand how the New York taxi service works.

The report describes step by step how our analysis is performed. It starts from data acquisition, compression, transformation and cleaning and then proceeds analyzing different aspects of the data in order to provide the reader with an overview of how the taxi service in New York works.

The main technology enabling the analysis of such dataset is Apache Spark, coupled with R for data visualization.

\section{Data Acquisition and compression}

The main dataset used for the analysis is publicly available at [URL]. We download the data for the time period from January 2010 to June 2018, both for Yellow and Green taxis which consists of ~200GB of comma separated value files containing over a 1 Billion records. The schema of the dataset varies by year and taxi company for a total of 6 different schemas. The main columns available through the datasets refer to pickup and dropoff datetimes, trip distance, number of passengers, fare amount, total amount, tip amount, extras and taxes, tolls, ratecode and payment type. The most notable difference between the data is the information about pickup and dropoff locations which is expressed with latitutes and longitudes until 2016 and with numerical taxi zone identifiers for the following years.

After acquisition of the data we proceed to compress it to .tar.gz format in order to handle them more effectively. The compression is significant, with a resulting dataset size of 43GB. The gzip format however is not particularly suited for big data analysis. The parquet file format provides significant performance advantages such as columnar storage, which allows computations that only need specific columns to access them selectively, specific compression algorithms for each column with dictionary specific encodings and the possibility to decode the file even partially which is convenient in distributed environments such as Spark. The size of the dataset is slightly reduced to 39GB.

\section{Data transformation}

The parquet dataset is then transformed into a common schema format to uniform the data and ease the analysis. We decide to adopt the following schema

XXXXXXXXXXXXXXX Describe one by one

\begin{itemize}
	\item taxi\_company
	\item pickup\_datetime
	\item dropoff\_datetime
	\item pickup\_location\_id
	\item dropoff\_location\_id
	\item passenger\_count
	\item trip\_distance
	\item ratecode\_id
	\item fare\_amount
	\item tolls\_amount
	\item total\_amount
	\item mta\_tax
	\item improvement\_surcharge
	\item extra
	\item tip\_amount
	\item payment\_type
\end{itemize}

Note that the common schema codifies the pickup and dropoff locations as the ids of the zone where the pickup or the dropoff happened and not as coordinates. The main challenge for schema conversion is the transformation of dropoff and pickup locations expressed as coordinates into the respective taxi zones. The dataset is accompanied by a shapefile which specifies the geographical boundaries of each zone. Unfortunately, naive algorithms for dataset conversion directly using the shapefile are able to only convert some tens of records per second per processor which would make the conversion of 1 Billion of records unfeasible. Instead, we decide to develop a more efficient algorithm for zone association which makes use of look up tables to increase the performance of the conversion up to the thousands of records per second, allowing the complete conversion of the dataset.

\subsection{Coordinates to Zone conversion algorithm}

Until 2016, information about pickup and dropoff locations is represented through latitudes and longitudes expressed in the World Geodetic System (WGS) coordinate system. The need to have a consistent schema of the data through the years, lead us to develop an efficient algorithm to convert the information about pickup and dropoff locations to numerical taxi zone identifiers. 

According to the shapefile provided with the dataset, the city of New York is divided in 265 zones and the geometric locations, representing each of these zones, are expressed through the NAD83 / New York Long Island (ftUS) coordinate reference system. Our first attempt is to divide New York in a 1000x1000 lookup matrix in which each tile is delimited by some coordinates and consists of a list of possible zones contained in these boundaries. This is achieved by including each shape record, representing a specific zone of New York, in the tiles forming the smallest rectangle delimiting the polygon constructed using the geometric shapes of that zone. Due to the lack of precision through which each zone is stored inside the tiles of the matrix, each tile may contain many zones.

Given the latitude and the longitude of a location, after converting the data to the same coordinate reference system used in the shapefile, the numerical taxi zone identifier can be obtained by looking at the corresponding tile inside the lookup matrix and retrieving the shape record containing that point. As the conversion of an entire dataset possibily involves the scan of all the zones contained in the target tiles, the algorithm is still inefficient. The optimization that allows us to speed up the procedure is to minimize the number of zones contained in each tile. This is achieved by defining a set of probing points, such as the ones represented in fig. \ref{fig:probingPoints}, for checking which of the zones are actually contained inside each tile. This allows us to create a refined lookup matrix. To further reduce the time of execution of the algorithm, every time it finds a tile with more than one zone, it randomly chooses one. 

The dataset resulting from the conversion of the coordinates has a size of 24GB.

\begin{figure}
	\centering
	\includegraphics{resources/probing_points.pdf}
	\caption{The probing points used for checking which zones are actually contained in a tile of the lookup matrix.}
	\label{fig:probingPoints}
\end{figure}

\section{Data cleaning}

An analysis of the dataset in the common format highlights data quality problems in some of its entries. To increase the quality of the analysis we decide to perform some data cleaning on the dataset.
Null entries are a minority of the dataset, so we decide to drop every entry in the dataset with a null attribute. Every categorical feature is then enforced to assume a value in its set of legal values. In particular, location ids are enforced to be valid ids and each entry for which our coordinate conversion algorithm was not able to identify a correspondent taxi zone is dropped. We then plot the distribution of the values of each numerical feature and conservatively identify a point in its tail after which all entries are dropped, for example tolls amounts greater than 120\$ are discarded because very irrealistic according to the distribution of the tolls and probably symptom of poor data quality. We also conservatively drop entries for which the duration is greater than 24 hours or for which the year is not in the range from 2010 to 2018.

The resulting dataset contains 1.07 Billion entries versus the original 1.38 Billion entries and the size of the dataset reduced to 20GB.

%1376582531 initial entries
%1077952992 final entries


\section{Preliminary analysis}
\label{sec:preliminaryAnalysis}
The first phase of the analysis is the study of the domain. The New York taxi service is operated by private individuals owning taxi licenses. Two kind of taxi cabs are present, each one with specific characteristics and limitations: yellow cabs, also called medallion taxis, and green cabs, also called boro taxis. The city of New York is divided in five different districts called boroughs, shown in \cref{fig:boroughsMap}: Manhattan, Bronx, Brooklyn, Queens and Staten Island and each one is further divided into taxi zones. Both yellow and green taxis share the same fare system, but, while yellow cabs are allowed to pickup passengers anywhere in the five boroughs, green cabs are not allowed to pick up passengers in South Manhattan, LaGuardia Airport and JFK Airport, the most profitable zones for yellow taxis, but are allowed to drop the passengers anywhere. The reasons for the limitations of green taxis are two. The first is that before the introduction of green taxis, finding a cab in the boroughs outside Manhattan was challenging because yellow taxis prefer to stay in the more profitable Manhattan zone, so this limitation favors the distribution of taxis in the outer zones. The second reason is that the introduction of new licenses for green taxis was seen as a menace from the yellow cab business which strongly opposed to their introduction and obtained that green taxis would not be allowed to pick up passengers at the lucrative LaGuardia and JFK airport zones unless previously arranged by passengers with the driver.

\begin{figure}
	\centering
	\includegraphics{resources/boroughsMap.jpg}
	\caption{The five boroughs of New York. Note the position of the three airports: LaGuardia, JFK and Newark.}
	\label{fig:boroughsMap}
\end{figure}

The fare system is shared by both yellow and green cabs. The amount starts from a base of 2.50\$ and increases as a function of time when moving slowly and as a function of distance when moving faster. Some extras may be added based on the hour of the day as well as various taxes. The passenger is also requested to pay for any incurred toll and is requested to tip the driver with a variable amount that ranges around 20\% of the total. Fares between any zone of Manhattan and LaGuardia airport have a special flat rate of 52\$ plus extras. Newark airport also havs a special rate which adds a 17.50\$ surcharge.

\begin{figure}
	\centering
	\includegraphics[width=1\columnwidth]{resources/fhv_vs_taxi.pdf}
	\caption{Historical taxi and ride hailing apps activity. Data from https://github.com/toddwschneider/nyc-taxi-data}
	\label{fig:uberVsTaxis}
\end{figure}

The analysis starts with a high level, exploration of the data. As shown in \cref{fig:travelsByYear} the number of total trips performed in the years grows slightly until 2015 and decreases in the following years. Note immediately that 2010 and 2016 are anomalous years due to data quality problems that caused the removal of entire months of entries for those year. Looking at the profits, calculated based on the total amount we can see an increasing trend in the years until 2015, after which there is a drop in the profits because of the reduced number of trips performed by the taxis. The growth of the profits, despite the only slight increase of the number of trips is given by a higher cost per trip in the years, as shown in \cref{fig:travelsByYear}. The decline of the number of taxi trips started in 2015 can be explained by the growth in popularity of ride hailing services such as Uber, Lyft and Juno which from 2015 are becoming widespread in the city as shown in \cref{fig:uberVsTaxis} and in 2017 surpassed taxis in popularity.

\begin{figure}
	\centering
	\includegraphics[width=1\columnwidth]{resources/base_plots/travels_by_year.pdf}
	\caption{Total trips performed each year. Data for years 2010 and 2016 is missing due to data quality problems, while data for 2018 is only relative for the period January-June. Note the decrease started in 2015.}
	\label{fig:travelsByYear}
\end{figure}

\begin{figure}
	\centering
	\includegraphics[width=1\columnwidth]{resources/base_plots/pickup_hour_dist.pdf}
	\caption{Number of pickups as a function of the pickup hour. Note the major activity periods at 8-16 and 20-24.}
	\label{fig:pickupHourDist}
\end{figure}

\begin{figure}
	\centering
	\includegraphics[width=1\columnwidth]{resources/base_plots/avg_speed_by_pickup_hour_dist.pdf}
	\caption{Average speed as a function of pickup hour. Note how the average speed increases in the periods of minor activity.}
	\label{fig:speedByHour}
\end{figure}

\begin{figure}
	\centering
	\includegraphics[width=1\columnwidth]{resources/base_plots/pickup_location_id_dist_map.pdf}
	\caption{Map in logarithmic scale showing the number of pickups per zone. Dropoff location map is omitted because of close similarity.}
	\label{fig:pickupDistributionMap}
\end{figure}

\begin{figure}
	\centering
	\includegraphics[width=1\columnwidth]{resources/base_plots/avg_speed_by_pickup_location_map.pdf}
	\caption{Map showing the average speed per pickup zone. Note how outer boroughs are typically feature higher average speeds, an indicator of the quantity of traffic.}
	\label{fig:speedMap}
\end{figure}

\begin{figure}
	\centering
	\includegraphics[width=1\columnwidth]{resources/base_plots/pickup_vs_dropoff.png}
	\caption{Map showing pickups points in green and dropoff points in red. Notice how dropoff points are granularly distributed across every street, while pickup points concentrate in Manhattan and on the main streets, meaning that people need to move to major roads in order to find taxis. Looking closely it can also be noted that the most blurry areas, South of Central Park and Southest part of Manhattan, which correspond to high GPS error area, are the ones which indeed have the highest concentration of tall buildings.}
	\label{fig:pickupDropoffImageMap}
\end{figure}

\begin{figure}
	\centering
	\includegraphics[width=1\columnwidth]{resources/base_plots/overall_pickups.pdf}
	\caption{Pickups count by day. Notice the decline in taxi pickups started in 2015.}
	\label{fig:overallPickups}
\end{figure}

The study continues with the analysis of pickups and dropoffs. \cref{fig:overallPickups} shows the number of pickups as a function of time. As already suggested by \cref{fig:travelsByYear} we notice a significant decrease in taxi service popularity starting from 2015. As depicted in \cref{fig:pickupHourDist} we can see that the periods of major activity for pickups are 8-16 and 20-24. The dropoffs follow the same pattern. Interestingly, by looking at the variation of the average speed through the day as shown in \cref{fig:speedByHour} we note that there is an inverse relationship between number of pickups and average speed, suggesting that in the highlighted time frames the city is subject to higher traffic congestions.
The most active pickup zones, as we can see from \cref{fig:pickupDistributionMap} are Manhattan, the part of Brooklyn closer to Manhattan, LaGuardia and JFK airports. The dropoff zones closely follow the pattern of the pickup zones, but are slightly more evenly distributed across all zones. \cref{fig:pickupDropoffImageMap} highlights the situation, showing that dropoff locations are more granularly distributed with respect to pickup locations. It must be noted however that the majority of the activity is concentrates in the most profitable Manhattan and airport sectors. \cref{fig:pickupDropoffImageMap} shows an  visualization of pickup and dropoff locations that highlights better the concentration of pickup activity in the zones noted above.
We also note that, as shown in \cref{fig:speedMap} the average speed per pickup location is inversely proportional to the number of pickups in that location, with the slowest zones being Manhattan, South Bronx and North Brooklyn. We can assume this zones to be the most subject to traffic.

\begin{figure}
	\centering
	\includegraphics[width=1\columnwidth]{resources/base_plots/avg_total_amount_by_pickup_location_map.pdf}
	\caption{Map showing the number the average total amount by pickup location. Note the inverse relationship with the distance from Manhattan.}
	\label{fig:totalAmountMap}
\end{figure}

\begin{figure}
	\centering
	\includegraphics[width=1\columnwidth]{resources/base_plots/avg_distance_by_pickup_location_map.pdf}
	\caption{Map showing the number the average trip distance by pickup location. Note the inverse relationship with the distance from Manhattan and the correlation with the average total amount map.}
	\label{fig:distanceMap}
\end{figure}

\begin{figure}
	\centering
	\includegraphics[width=1\columnwidth]{resources/base_plots/avg_total_amount_by_pickup_hour.pdf}
	\caption{Average total amount divided by pickup hour. Note the two peaks at 5 and 16.}
	\label{fig:totalAmountByHour}
\end{figure}

\begin{figure}
	\centering
	\includegraphics[width=1\columnwidth]{resources/base_plots/avg_total_amount_by_month.pdf}
	\caption{Average total amount divided by month. May and September-October result the most profitable periods.}
	\label{fig:totalAmountByMonth}
\end{figure}

It is also interesting to notice how total amounts relate to the pickup location. As depicted in \cref{fig:totalAmountMap} it can be noted that there is an inverse relationship between average total amount and zone distance from Manhattan. This is an expected outcome of the fact that Manhattan is one of the favourite dropoff locations, so on average the furthest zones from it are the most expensive. \cref{fig:distanceMap} shows average trip distance by pickup location. Here we can note the direct correlation with the average total amount of the preceding picture. The average total amount also varies during the day, as depicted in \cref{fig:totalAmountByHour}. Two different peak periods are identified around 5 and 16. The discontinuity at 16 is caused by the introduction of a 1\$ extra from 16 to 20 as a standard component of the fare amount, while the peak at 5 is caused by a sharp increase of the average trip distance from 2.92 to 4.61 miles due to an increased proportion of people going to the airport early in the morning with respect to the normal traffic. It can also be noted from \cref{fig:totalAmountByMonth} that the average total amount varies during the months, with most profitable periods being May and September-October.

\begin{figure}
	\centering
	\includegraphics[width=1\columnwidth]{resources/base_plots/avg_total_amount_by_year.pdf}
	\caption{Average total amount by year. In 2012 the New York taxi commission increased the fare rates.}
	\label{fig:totalAmountByYear}
\end{figure}

\begin{figure}
	\centering
	\includegraphics[width=1\columnwidth]{resources/base_plots/payment_type_distr.pdf}
	\caption{Distribution of payment types during the years. Note the steady decline of cash payments in favor of credit card payments.}
	\label{fig:paymentTypeByYear}
\end{figure}

\begin{figure}
	\centering
	\includegraphics[width=1\columnwidth]{resources/base_plots/fare_amount_distr.pdf}
	\caption{Distribution of fare amounts. Note the peaks at 45\$ and 52\$ corresponding to the JFK airport flat rates before and after the 2012 fares revision operated by the New York taxi commission.}
	\label{fig:fareAmountDistr}
\end{figure}

\begin{figure}
	\centering
	\includegraphics[width=1\columnwidth]{resources/base_plots/overall_fare_amount.pdf}
	\caption{Average fare amount by day in \$. Note how in 2012 the new fare system caused a prominent increase in fare rates.}
	\label{fig:overallFareAmount}
\end{figure}

We also noted a general increase of the total amount during the years as noted in \cref{fig:totalAmountByYear} which is a consequence of many factors. First, the average fare amount passed from 10.18\$ of 2010 to 12.61\$ of 2018 due to a general revision of fare rates operated by the New York taxi commission in 2012 which can be seen in \cref{fig:overallFareAmount}. Other factors include slight increases in the average tolls, extras and improvement surcharge during the years. It must be noted however that the increase is exacerbated by the fact that total amounts in our datasets include tips only for trips paid with credit card and that during the years there has been a constant increase in the use of credit cards as shown in \cref{fig:paymentTypeByYear}. If our dataset included also cash tips, then we would still notice an increase in the total amount during the years, although more subtle.
The overall distribution of fare amounts is shown in \cref{fig:fareAmountDistr}. Note the sharp discontinuities at 52\$ and 45\$. These represents the special fare rate of trips linking Manhattan to the JFK airport before and after a general revision of fare rates operated by the New York taxi commission during 2012.

A consistent part of the total amount is given by tips. A calculation of average tips percentages, calculated only on credit card paid trips, shows that the typical trip is tipped 18.6\% and this rate remains stable in all the analyzed years. This value corresponds to the guidelines of many online sources that suggest tipping taxi drivers around 15 to 20\% of the total amount.

\begin{figure}
	\centering
	\includegraphics[width=1\columnwidth]{resources/base_plots/trip_distance_distr.pdf}
	\caption{Distribution of trip distances. Note the three modes at 1, 9 and 17 miles.}
	\label{fig:tripDistanceDistr}
\end{figure}

Other interesting findings regard the distribution of trip distances which is shown in \cref{fig:tripDistanceDistr}. It can be observed that the distribution is trimodal, with the first mode at 1 mile which represents short trips performed mainly inside Manhattan, the second at 9 miles, which represents trips between Manhattan and LaGuardia airport, the third at 17 miles, which represents trips between Manhattan and JFK airport.

\begin{figure}
	\centering
	\includegraphics[width=1\columnwidth]{resources/base_plots/passenger_count_dist.pdf}
	\caption{Distribution of passenger counts.}
	\label{fig:passengerCountDistr}
\end{figure}

Looking at the passenger count distribution in \cref{fig:passengerCountDistr} we can note that the majority of trips is performed with 1 or 2 passengers, while there is also a component of group rides which causes a peak at 5 passengers.

By looking at the distribution of payment types during the years in \cref{fig:paymentTypeByYear} it can be noted a strong trend towards paying with credit card versus cash, which was prevalent until 2012.

\begin{figure}
	\centering
	\includegraphics[width=1\columnwidth]{resources/base_plots/tolls_amount_distr.pdf}
	\caption{Distribution of tolls amount. Note the modes at 0\$ and 5\$.}
	\label{fig:tollsAmountDistr}
\end{figure}

As a last general remark we note that the distribution of tolls amount as shown in \cref{fig:tollsAmountDistr} is bimodal, with the first mode at 0\$ and a smaller mode at 5\$, representing the fact that the majority of trips are not subject to tolls, while the one that are pay on average 5\$ of tolls. These corresponds to tolls for Queens-Midtown Tunnel, Brooklyn-Battery Tunnel and the Triboro Bridge, which are convenient ways to avoid traffic for reaching the various parts of Manhattan from Brooklyn and Queens.

\section{Traffic segmentation}

After the data cleaning and preliminary analysis phase, the identification of the different categories of trips is achieved through k-means clustering. In order to prepare the data for the building model phase, the following steps are performed:

\begin{itemize}
	\item the string column containing the taxi company is encoded to a column of label indices named taxi\_company\_indexed\_encoded
	\item the pickup and dropoff datetime are encoded to 3 new features: pickup\_hour, dropoff\_hour, weekend
	\item from the information about the distance, the pickup and the dropoff datetime of the trip, the speed property is extracted
	\item a one-hot encoder is used to map each of the following columns of category indices: taxi\_company\_indexed\_encoded, ratecode\_id and payment\_type to a column of binary vectors, with at most a single one-value per row that specifies the input category index
	\item the different features resulted from the one-hot encoding of the previous step and other raw features are combined into a single feature vector called partial\_features
	\item the different features generated by the previous feature transformers (except one-hot encoded features) and other raw features are combined into a single feature vector called unscaled\_features\_vector
	\item the unscaled\_features\_vector is scaled to have unit standard deviation and zero mean features. The resulting feature is called scaled\_features\_vector
	\item partial\_features and scaled\_features\_vector are combined into a single feature vector called clustering\_features that is used as feature vector for the k-means clustering algorithm
\end{itemize}  

The columns containing the ids of the zone where the pickup or the dropoff happened are not included in the building model phase because we do not want the model to use them as distinguishing criteria. The objective is to find the different types of traffic of the entire city, not the types of traffic of a certain zone.

The appropriate number of clusters for the k-means clustering algorithm is detected using the Elbow method. For our dataset, the number of clusters chosen is 4. By looking at \cref{fig:FareAmountDistr} and \cref{fig:TripDistanceDistr} it can be noted that the clustering is done according to fare\_amount and trip\_distance features. More precisely, the group 0 identifies low fare amounts and very small distance trips, the group 1 identifies very high fare amount and long distance trips, the group 2 identifies high fare amount and medium distance trips and the group 3 identifies moderate fare amount and small distance trips. Further analysis show that the average speed differs from one cluster to another, it is higher for long distance trips than for small distance trips. In addition, tips are on average higher for trips of group 0 than for the other types of trips. An analysis of the types of payments through the years per cluster, highlights that trips of group 1 are paid with higher percentage of cases with credit cards than trips of other groups. Moreover, while expensive trips remain costant through the week, inexpensive trips increase during the weekend.

\begin{figure}
	\centering
	\includegraphics[width=1\columnwidth]{resources/clustered_plots/fare_amount_distr.pdf}
	\caption{Distribution of fare amounts per cluster.}
	\label{fig:FareAmountDistr}
\end{figure}

\begin{figure}
	\centering
	\includegraphics[width=1\columnwidth]{resources/clustered_plots/trip_distance_distr.pdf}
	\caption{Distribution of trip distances per cluster.}
	\label{fig:TripDistanceDistr}
\end{figure}

\cref{fig:TrafficGraphWithLegend} shows the network graph constructed for better visualize the different types of traffic inside the city of New York. Each node represents one of the 265 zones of New York, while each edge represents a traffic flow from a starting zone to a destination zone. The colour of the nodes depend on the borough to which they belong, while the colour of the edges depend on the cluster to which they belong. An important aspect to note is that the number of edges starting from a node is variable and depends on the number of pickups per that zone. In addition, each node has a set of attributes, whose value can be LOW, MEDIUM and HIGH, describing the level of activity, the level of distance, the level of duration, the level of speed and the level of total amount of that zone. For each of these attributes, the values are calculated by finding the first and third quartiles of the corresponding distribution. The value of a given attribute for a node is considered LOW if the value of the node lies below the 25\% of the data in the dataset, MEDIUM if the value of the node lies between the 25\% and the 75\%, and HIGH if the value of the node lies above the 75\% of the data in the dataset. By looking at the graph it can be noted that the clustering found as group 1 the traffic to the JFK airport and as group 2 the traffic to the LaGuardia airport. For a better visualization of the network graph, please refer to the interactive version of the graph and its legend that can be found on traffic\_graph.zip. 

\begin{figure}
	\centering
	\includegraphics[width=1\columnwidth]{resources/traffic_graph_with_legend.png}
	\caption{Network graph showing the different types of traffic found using k-means clustering.}
	\label{fig:TrafficGraphWithLegend}
\end{figure}

\section{Traffic flow analysis}

One of the objectives of this report is to better understand how people move inside the city of New York. We partially describe this activity in \cref{sec:preliminaryAnalysis} by showing the high activity levels of Manhattan and of the Airport zones. In order to better understand the flows of traffic we decide to plot for each pickup borough, the distribution of dropoff locations as a function of time.

\begin{figure}
	\centering
	\includegraphics[width=1\columnwidth]{resources/base_plots/bronx_dropoff_location_id_dist_map.pdf}
	\caption{Map in logarithmic scale showing the preferred dropoff locations for rides starting inside the Bronx borough.}
	\label{fig:bronxDropoffMap}
\end{figure}

\cref{fig:bronxDropoffMap} shows the distribution of dropoff locations for pickups happened in the Bronx borough. The majority of dropoffs as expected happen inside the Bronx borough itself, North Manhattan or the two airports.

\begin{figure}
	\centering
	\subfloat[Overall dropoff locations (log scale)]{\label{fig:brooklynDropoffMap}\includegraphics[width=1\columnwidth]{resources/base_plots/brooklyn_dropoff_location_id_dist_map.pdf}}\\
	\subfloat[Dropoff locations at 9]{\label{fig:brooklyn9DropoffMap}\includegraphics[width=1\columnwidth]{resources/base_plots/brooklyn_dropoff_location_id_by_pickup_hour_dist9_map.pdf}
	}
	\caption{Map showing the preferred dropoff locations for rides starting inside the Brooklyn borough, overall and at 9. Notice the high concentration of dropoffs in the Tribeca Manhattan district at 9.}
\end{figure}


\cref{fig:brooklynDropoffMap} shows the distribution of dropoff locations for pickups happened in the Brooklyn borough. Also in this case the majority of dropoffs happen inside the borough itself, the two airports or the adjacent Manhattan zones. An interesting consideration can be done by looking at \cref{fig:brooklyn9DropoffMap} which shows the favourite dropoff locations for rides beginning at 9. From 7 to 9 we can see a high rates of dropoffs in the Tribeca district of Manhattan, which is the ending point of the Brooklyn Bridge. This flow vanishes during the day and is probably caused by workers moving into the city early in the morning. 

\begin{figure}
	\centering
	\subfloat[Overall dropoff locations (log scale)]{\label{fig:manhattanDropoffMap}\includegraphics[width=1\columnwidth]{resources/base_plots/manhattan_dropoff_location_id_dist_map.pdf}}\\
	\subfloat[Dropoff locations at 10]{\label{fig:manhattan10DropoffMap}\includegraphics[width=1\columnwidth]{resources/base_plots/manhattan_dropoff_location_id_by_pickup_hour_dist10_map.pdf}}\\
	\subfloat[Dropoff locations at 22]{\label{fig:manhattan33DropoffMap}\includegraphics[width=1\columnwidth]{resources/base_plots/manhattan_dropoff_location_id_by_pickup_hour_dist23_map.pdf}}
	\caption{Map in showing the preferred dropoff locations for rides starting inside the Manhattan borough, overall, at 10 and at 23.}
	
\end{figure}

\cref{fig:bronxDropoffMap} shows the distribution of dropoff locations for pickups happened in the Manhattan borough. Most dropoffs happen within Manhattan itself or in the airports. During the day the most active dropoff zones are those on the South East corner of Central Park, while after 20, most of the dropoffs concentrate in the South West Manhattan area and in East Village. From this fact and from some research on the New York nightlife we infer that this shift is caused by people moving to these zones to experiment the New York nightlife. The fact is cross checked by looking at the locations where most pickups happen at night, which are again South West Manhattan and East Village.

\begin{figure}
	\centering
	\subfloat[Overall dropoff locations (log scale)]{\label{fig:queensDropoffMap}\includegraphics[width=1\columnwidth]{resources/base_plots/queens_dropoff_location_id_dist_map.pdf}}\\
	\subfloat[Dropoff locations at 12]{\label{fig:queensDropoff12Map}\includegraphics[width=1\columnwidth]{resources/base_plots/queens_dropoff_location_id_by_pickup_hour_dist12_map.pdf}}
	\caption{Map showing the preferred dropoff locations for rides starting inside the Queens borough, overall and at 12. Note how at 12 a consistent part of the traffic goes to the Manhattan district under Central Park. This movement remains constant all day from 7 to 16.}
\end{figure}

\cref{fig:queensDropoffMap} shows the distribution of dropoff locations for pickups happened in the Queens borough. The majority of dropoffs happen in North West Queens, the adjacent part of Manhattan and at the airports. We notice that from 7 to 16 there is a steady movement of people from Queens to the Manhattan districts South of Central Park, similar to that shown in \cref{fig:queensDropoff12Map}. By isolating the airport traffic it can be seen that a part of this phenomenon is caused by trips started at airports that seem to prefer the zones below Central Park as destination.

\begin{figure}
	\centering
	\includegraphics[width=1\columnwidth]{resources/base_plots/staten_island_dropoff_location_id_dist_map.pdf}
	\caption{Map in logarithmic scale showing the preferred dropoff locations for rides starting inside the Staten Island borough.}
	\label{fig:statenIslandDropoffMap}
\end{figure}

\cref{fig:statenIslandDropoffMap} shows the distribution of dropoff locations for pickups happened in the Staten Island borough. This district features an anomalous behavior in that the majority of dropoffs happen in the districs West of Central Park and the airports instead of inside Staten Island itself. This behavior remains constant throughout the whole day.

\section{Yellow vs Green cabs}

A major change in the taxi business was given by the addition of green taxis in addition to the traditional yellow cabs. In this section we analyze the impact of this introduction and the difference from the yellow taxis.

\begin{figure}
	\centering
	\includegraphics[width=1\columnwidth]{resources/base_plots/travels_by_year_and_company.pdf}
	\caption{Total trips performed each year divided by taxi type. Data for years 2010 and 2016 is missing due to data quality problems, while data for 2018 is only relative for the period January-June.}
	\label{fig:travelsByYearAndCompany}
\end{figure}

\cref{fig:travelsByYearAndCompany} shows the distribution of trips between the companies in the years from 2010 to 2018. We can notice that the introduction of green taxis caused a slight reduction of the trips performed by yellow taxis and from 2015 both types of taxi suffered from a substantial decrease due to the competition of companies such as Uber.

\begin{figure}
	\centering
	\includegraphics[width=1\columnwidth]{resources/base_plots/fare_amount_by_company_distr.pdf}
	\caption{Fare amount distribution divided by company. 45\$ and 52\$ fares represent special fare rates to the airport.}
	\label{fig:fareAmountByCompany}
\end{figure}

Looking at \cref{fig:fareAmountByCompany}, which compares the distribution of fare amounts we immediately notice that the curves have similar shape, although the longest and profitable airport trips, the ones with 52\$ fare rate, are almost always performed by yellow cabs. This is expected, since green cabs can only perform dropoffs at the airport and pick up passengers only if the trip is arranged in advance, which is a severe limitation.

\begin{figure}
	\centering
	\includegraphics[width=1\columnwidth]{resources/base_plots/trip_distance_by_company_distr.pdf}
	\caption{Trip distance distribution divided by company.}
	\label{fig:tripDistanceByCompany}
\end{figure}

By analyzing \cref{fig:tripDistanceByCompany} we also note that, while the distribution of yellow taxis is trimodal, the distribution for green taxis is not. In particular we notice that green taxis miss the 9 miles and 17 miles mode, which are the trips to the JFK and LaGuardia airports. Trips at LaGuardia airport do not happen often for the same limitations of green taxis explained in the former paragraph.


\begin{figure}
	\centering
	\includegraphics[width=1\columnwidth]{resources/base_plots/pickup_location_by_company_after_2014_distr_difference_map.pdf}
	\caption{Map showing which taxi type preveals in the various zones since year 2014. Yellow zones show where yellow taxis perform the most pickups, green zones represent the ones where green taxis perform more pickups. Notice how Manhattan and the airport zones are prevalently served by yellow taxis.}
	\label{fig:companyDivisionMap}
\end{figure}
\begin{figure}
	\centering
	\includegraphics[width=1\columnwidth]{resources/base_plots/yellow_vs_green_pickup.png}
	\caption{ view showing pickup points of yellow and green taxis in their respective color. Note how Manhattan and the airports feature yellow prevalence, while the other zones feature a green prevalence.}
	\label{fig:companyDivisionImageMap}
\end{figure}

Green taxis were introduced explicitly to help serving the outer boroughs, where it is difficult to find a yellow cab, because they prefer to remain in the profitable Manhattan zone. \cref{fig:companyDivisionMap} shows which taxi type is more popular in the various zones since year 2014. Indeed we can see that green taxis became prevalent in North Manhattan, where they are allowed to pick up passengers, and in every outer borough with the exception of Staten Island and the two airport zones where they are not allowed to perform non arranged pickups. \cref{fig:companyDivisionImageMap} shows an  view of the yellow-green taxi presence in New York.

\section{Airport traffic}

Given the relevance of the airport traffic emerged by the previous analysis step we decide to perform a more thorough analysis of the traffic originated or directed to the three airports. The first remark is that, due to Newark airport not being in a valid taxi zone, the data cleaning phase removed many of the trips to the Newark airpot and so data regarding it is not reliable.

\begin{figure}
	\centering
	\includegraphics[width=1\columnwidth]{resources/airport/airport_distr.pdf}
	\caption{Total number of trips performed from January 2010 to June 2018 divided by airport. Airport trips make 5.92\% of total taxi trips.}
	\label{fig:airportDistr}
\end{figure}

\begin{figure}
	\centering
	\includegraphics[width=1\columnwidth]{resources/airport/total_amount_by_airport.pdf}
	\caption{Total revenue from January 2010 to June 2018 divided by airport. Airport trips make 18.7\% of total taxi revenue.}
	\label{fig:airportProfitDistr}
\end{figure}

\begin{figure}
	\centering
	\includegraphics[width=1\columnwidth]{resources/airport/fare_amount_distr.pdf}
	\caption{Distribution of fare amount. Note the mode at ~30\$ corresponding to trips to LaGuardia airport and the 45\$ and 52\$ flat rate amounts corresponding to trips to JFK from Manhattan. The right heavy tail is given by trips to the JFK from outside Manhattan that are not subject to the flat rate.}
	\label{fig:airportFareAmount}
\end{figure}

We first notice from \cref{fig:airportDistr} that airport trips make a relatively small 5.92\% of the total number of taxi trips. Despite that, as \cref{fig:airportProfitDistr} shows, they are a major stream of revenues for taxi due to the very high per cost per trip making 18.7\% of their profits. In fact, by looking at the fare rate distribution in \cref{fig:airportFareAmount}, and comparing it with \cref{fig:fareAmountDistr} we can see that fare amounts for LaGuardia averages around 30\$, fare amounts towards JFK have a fixed fare rate of 52\$ (45\$ before fare rate revision operated in 2012), while the in majority of common taxi trips the fare amounts are below 10\$. It it thus not surprising that yellow taxi owners fought for forbidding green taxis to pick up passengers at airports.

\begin{figure}
	\centering
	\includegraphics[width=1\columnwidth]{resources/airport/payment_type_distr.pdf}
	\caption{Distribution of payment methods from 2010 to 2018. Note how already from 2011 credit card payments make for the majority of payments. The higher percentage of credit card payments is probably given by the high cost of airport trips.}
	\label{fig:airportPaymentType}
\end{figure}

By looking at the distribution of payment types in \cref{fig:airportPaymentType} it is also interesting to notice that the percentage of people paying with credit cards is higher than its percentage in common taxi trips shown in \cref{fig:paymentTypeByYear}. This is probably due to the higher total amounts.

\begin{figure}
	\centering
	\includegraphics[width=1\columnwidth]{resources/airport/pickup_hour_dist.pdf}
	\caption{Number of pickups divided by hour. The peak activity is reached at 14 and is almost absent from 24 to 5.}
	\label{fig:airportPickupHourDistr}
\end{figure}


From \cref{fig:airportPickupHourDistr} we can look at the number of trips as a function of pickup hour. This can be seen as a rough indicator of airport activities through the day. We notice that activity starts at about 5, then peaks at 14 and slowly decreases until 24, after which activity is enormously reduced. By comparing it with standard taxi activity shown in \cref{fig:pickupHourDist} we can see that airport traffic completely lacks the peak experimented at 20 and reinforces the hypothesis that the peak at 20 in normal traffic is caused by nightlife.

\begin{figure}
	\centering
	\includegraphics[width=1\columnwidth]{resources/airport/tolls_amount_distr.pdf}
	\caption{Distribution of tolls amount. Notice how the number of trips subject to tolls are the majority.}
	\label{fig:airportTollsAmount}
\end{figure}

As a last remark, by looking at the distribution of tolls amount in \cref{fig:airportTollsAmount} we notice that the number of trips subject to tolls is the majority with respect to the total number of trips. By comparing it with \cref{fig:tollsAmountDistr} we see that airport trips are responsible for the majority of trips that were subject to tolls.

\section{Conclusions}
This paragraph will end the body of this sample document.
Remember that you might still have Acknowledgments or
Appendices; brief samples of these
follow.  There is still the Bibliography to deal with; and
we will make a disclaimer about that here: with the exception
of the reference to the \LaTeX\ book, the citations in
this paper are to articles which have nothing to
do with the present subject and are used as
examples only.


%
% The following two commands are all you need in the
% initial runs of your .tex file to
% produce the bibliography for the citations in your paper.
\bibliographystyle{abbrv}
\bibliography{main}


\balancecolumns
\appendix
%Appendix A
\section{Headings in Appendices}
First appendix
\end{document}
